\documentclass{amsart}

\usepackage{setspace}
\begin{document}

\spacing{1.3}

\section*{Ovals Problem is not affine-invariant}

Take an affine transformation \(\gamma \mapsto A \gamma + b\). The \(b\) doesn't enter into the problem. Let's write \(\bar{\gamma} = A \gamma\). Define
\[
c = \|A\gamma'|/\|\gamma'\|
\]

Note that \(c\) is non-constant along the curve in general! In the particular case \(A \in O(2)\), \(c \equiv 1\).

Then 
\begin{align}
d\bar{s} &= c ds \\
\partial_{\bar{s}} &= (1/c) \partial_s \\ 
\bar{\kappa} &= (1/c^3) \kappa
\end{align}.

Incidentally, these relations lead to the affine-invariance of \(\int \kappa^{1/3} ds\).

Now look at the Rayleigh quotient:
\[
\int (f_{\bar{s}})^2 + \bar{\kappa}^2 f^2 d\bar{s} = \int \left(\frac{1}{c^2} (f_s)^2 + \frac{1}{c^6} \kappa^2 f^2\right) c ds = \int \frac{1}{c} (f_s)^2 + \frac{1}{c^5} \kappa^2 f^2 ds
\]

The scaling of \(c\) is all wrong here and it doesn't seem possible to make this work unless \(c \equiv 1\). 

Therefore the family of ovals does not correspond to a 1-parameter subgroup of affine transformations! The problem is if these transformations are required to preserve length, then it seems they must be orthogonal transformations, but then you can't get a family of ellipses this way!

Maybe, it's possible to transform \(f\) as well to make this work, say by initially normalising \(f\) to have \(\int f^2 = 1\) and then allow our affine transformations to vary length, but adjust for this in scaling \(f\) accordingly? Still doesn't seem likely to work...




\end{document}